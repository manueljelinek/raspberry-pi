%\documentclass[12pt,a4paper]{article}% Walter Schmidt
%\usepackage[T1]{fontenc}\usepackage{textcomp}
%\usepackage{mathpazo}
%\usepackage{courier}
%\usepackage{geometry,url,ngerman}



\parindent0pt


\newcounter{leaves}
\newcounter{directories}

\newenvironment{directory}[2][\linewidth]%
% Startet neues Verzeichnis 
% und produziert eine Minipage der angeg. Breite.
% Syntax: \begin{directory}[width]{text}
% text muss in eine \parbox der angegebenen Breite passen;
% wenn keine Breite angegeben ist, wird \linewidth angenommen.
{%
\setcounter{leaves}{0}%
\addtocounter{directories}{1}
\edef\directoryname{D\thedirectories}
\begin{minipage}[t]{#1}% <-------- !!!
  \setlength{\parindent}{\linewidth}
  \addtolength{\parindent}{-\dirshrink\parindent}
  \parskip0pt%
  \noindent
  \Rnode[href=-\dirshrink]{\directoryname}{\parbox[t]{#1}{#2}}%
  \par
}  
{\end{minipage}}

% !!! --> Problem:  
% Wegen [t] stimmt der Zeilenabstand _nach_ der minipage nicht.
% Der Referenzpunkt eines Knoten muss aber in der _ersten_ Zeile 
% liegen, mehrzeilige Knoten, also Unterverzeichnisse, mit ihrer
% ersten Zeile im Dateibaum verankert weren.

\newcommand{\file}[2][]{%
% Fuer einen einzelnen Eintrag innerhalb der directory-Umgebung.
% Das Argument darf seinerseits eine directory-Umgebung sein.
  \addtocounter{leaves}{1}%
  \edef\leaflabel{L\theleaves\directoryname}%
  \par
  \Rnode{\leaflabel}{\parbox[t]{\dirshrink\linewidth}{#2\hfill#1}}%
  \ncangle[angleA=270,angleB=180,armB=0,nodesep=1pt]
    {\directoryname}{\leaflabel}%
  % \typeout{\directoryname,\leaflabel}% Debugging
\par}

\newcommand{\dirshrink}{.95} 
% relative Verringerung der Breite der Verzeichniseintraege 
% pro Stufe


%\begin{document}

\begin{postscript}
\begin{directory}{\url{./}}
\file{\begin{directory}{\url{shell_only/}}
  \file{\begin{directory}{\url{source/}}
    \file{\begin{directory}{\url{include/}}
    \file{\url{*.h}}
    \end{directory}}
    \file{\url{init.s}}
    \file{\url{*.c}}
  \end{directory}}
\file{\url{kernel.ld}}
\file{\url{Makefile}}
\end{directory}}
\file{\begin{directory}{\url{framebuffer/}}
  \file{\begin{directory}{\url{source/}}
    \file{\begin{directory}{\url{include/}}
    \file{\url{*.h}}
    \end{directory}}
    \file{\url{init.s}}
    \file{\url{font.bin}}
    \file{\url{*.c}}
  \end{directory}}
\file{\url{kernel.ld}}
\file{\url{Makefile}}
\end{directory}}
\file{\begin{directory}{\url{raspbootin/}}
  \file{\begin{directory}{\url{raspbootin/}}
         \end{directory}}

    \file{\begin{directory}{\url{raspbootcom/}}
    \end{directory}}
    \file{\url{Makefile}}
\end{directory}}
\end{directory}
\end{postscript}

%\end{document}

