%\documentclass[a4paper,10pt]{article}
\documentclass[a4paper,10pt]{scrartcl}





\usepackage{geometry}
\usepackage{pdflscape}
\usepackage[utf8]{inputenc}
%\usepackage{graphicx}
%\usepackage{ngerman}
\usepackage[german]{babel}
\usepackage{epstopdf}
\usepackage{subfigure} 
\usepackage[pdftex]{graphicx}
\setlength{\parindent}{0pt}
\usepackage{listings}
\title{Dokumentation Shell - Framebuffer}
\author{Manuel Jelinek Martin Erb}
\date{\today}


\pdfinfo{%
  /Title    (Dokumentation Shell - Framebuffer)
  /Author   (Manuel Jelinek, Martin Erb)
  /Creator  (Manuel Jelinek, Martin Erb)
}

\begin{document}
\maketitle
\section*{Compiler}
Damit unser Makefile richtig funktioniert muss folgender Befehl ausgeführt
werden welcher den richtigen Compiler für das Raspberry installiert.
\begin{verbatim} 
sudo apt-get install gcc-arm-linux-gnueabihf 
\end{verbatim}

\section*{Shell}
Damit die Shell startet und richtig initialisiert wird muss folgende
Funktionen aufgerufen werden:
\begin{verbatim} 
shell();
\end{verbatim}
Fünf verschiedene Befehle sind in der Shell fix hardcoded implementiert:
exit, shutdown, restart, load, help.
Was diese Befehle machen kann man nachlesen wenn man die Shell startet und
help eingibt.
Der Befehl shutdown funktioniert nur wenn auch unser Raspbootcom verwendet wird
da wir dort kleine Änderungen vorgenommen
haben und somit vom Raspberry aus, Raspbootcom beenden können.\\
Der nützlichste Befehl ist der load-Befehl. Wenn das Programm einmal auf dem
Raspberry ausgeführt wird kann man am
PC durch eine zweite Konsole das kernel.img neu erstellen und mit dem Befehle
(load) neu auf das Raspberry kopieren und 
das Programm startet automatisch neu.
Um die Shell möglichst dynamisch zu halten kann man über die Funktion
\begin{verbatim} 
void addNewCommand(fcn_ptr function_pointer, char command_name, char* help_text)
\end{verbatim}
neue Befehle hinzufügen. Der erste Parameter ist der Pointer auf die Funktion
welche beim Aufrufen des Befehls ausgeführt werden soll.
Der zwei Parameter ist der Name des Befehls. Der dritte Parameter ist ein char
array im welchen der Text steht welche beim aufrufen der Funktion
help angezeigt werden soll. Falls der User keinen Hilfe Text anzeigen möchte
dann kann muss dieser Pointer auf NULL zeigen.
Ein Beispiel dieser Funktion kann man in der shell.c Zeile 21 sehen.

\newpage
\section*{Framebuffer}
Initialisieren des Framebuffer am Raspberry Pi:
Nur für die Initialisierung des Framebuffers werden folgende Files benötigt:
\begin{verbatim} 
mailbox.(h/c)
memutils.(h/c)
framebuffer.(h/c)
\end{verbatim}
Dabei muss nur die File \texttt{framebuffer.h} inkludiert werden. Die beiden 
anderen Files werden zusätzlich vom framebuffer intern verwendet. Um den
Framebuffer zu initialisieren, wird entweder die Funktion
\begin{verbatim} 
fbInitNativ();
\end{verbatim}
verwendet, die den Framebuffer mit der maximalen Auflösung des
Monitors mit einer
Farbtiefe von 16bit initialisiert oder man verwendet die Funktion
\begin{verbatim} 
fbInit(size_x, size_y, color_depth);
\end{verbatim}
Dieser kann man eine beliebige Auflösung übergeben und als dritten Parameter
noch die Farbtiefe.
Dabei sind die Farbmodi 16bit, 24bit und 32bit möglich.
Für den gewünschten Modus gibt es im Header-File entsprechende Defines:
\begin{verbatim} 
COLORMODE_16BIT
COLORMODE_24BIT
COLORMODE_32BIT
\end{verbatim}



Die minimale Auflösung des Framebuffer ist mit 640x480 Pixel und die maximale
mit der des Monitors
festgelegt. Wird einer der Werte überschritten bzw. unterschritten, so wird der
jeweilige Wert auf
das entsprechende maximum oder minimum gesetzt.

Beispiel:
\begin{verbatim} 
fbInit(1024, 768, COLORMODE_16BIT);
\end{verbatim}
Nach der Initialisierung werden Zeichen in weiß auf einem schwarzen Hintergrund
dargestellt. Um
diese Farben zu ändern, werden folgende Funktionen verwendet:
\begin{verbatim} 
consoleForegroundColor(new_color);  // ändert die Zeichenfarbe
consoleBackgroundColor(new_color);  // ändert die Hintergrundfarbe
\end{verbatim}
Einige Standardfarben wurden im Header-File definiert. Dabei sind die Defines
vom 32bit Modus
ebenfalls für den 24bit Modus zu verwenden (Es wird nur der Alpha-Kanal
ignoriert).
Beispiele:
\begin{verbatim} 
consoleForegroundColor(COLOR16_RED);
consoleForegroundColor(COLOR32_RED);
consoleBackgroundColor(COLOR16_BLUE);
consoleBackgroundColor(COLOR32_BLUE);
\end{verbatim}
Mit der Funktion 
\begin{verbatim} 
consoleWriteChar(character)
\end{verbatim}
können einzelnen Zeichen in den
Framebuffer geschrieben
werden. Dabei wird der Standard ASCII Zeichensatz verwendet und um weitere
printbare Zeichen der 'CodePage 437' erweitert (Werte $>$ 127). Des weiteren
setzt die Funktion die Position, an die das Zeichen
geprintet wird auf den nächsten Block weiter bzw. führt am ende der Zeile einen
Zeilenumbruch durch.
Ist das Ende des Framebuffers erreicht, so werden alle Zeichen um eine Zeile
nach oben verschoben und die letzte Zeile gelöscht.
Mit der Funktion
\begin{verbatim} 
clearScreen()
\end{verbatim}
kann der Framebuffer wieder gelöscht werden. Des weiters wird der Cursor wieder
an die linke obere Position des Monitors gesetzt.
\end{document}
